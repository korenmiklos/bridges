%2multibyte Version: 5.50.0.2890 CodePage: 1252

\documentclass[12pt]{article}
\usepackage{amsmath,amsthm,amsfonts}
\usepackage[utf8]{inputenc}
\usepackage{graphicx}
\usepackage[onehalfspacing]{setspace}

%TCIDATA{OutputFilter=LATEX.DLL}
%TCIDATA{Version=5.50.0.2890}
%TCIDATA{Codepage=1252}
%TCIDATA{<META NAME="SaveForMode" CONTENT="1">}
%TCIDATA{BibliographyScheme=Manual}
%TCIDATA{Created=Sunday, February 17, 2013 16:58:07}
%TCIDATA{LastRevised=Sunday, April 28, 2013 17:27:32}
%TCIDATA{<META NAME="GraphicsSave" CONTENT="32">}
%TCIDATA{<META NAME="DocumentShell" CONTENT="Standard LaTeX\Blank - Standard LaTeX Article">}
%TCIDATA{CSTFile=40 LaTeX article.cst}

\newtheorem{theorem}{Theorem}
\newtheorem{acknowledgement}[theorem]{Acknowledgement}
\newtheorem{algorithm}[theorem]{Algorithm}
\newtheorem{axiom}[theorem]{Axiom}
\newtheorem{case}[theorem]{Case}
\newtheorem{claim}[theorem]{Claim}
\newtheorem{conclusion}[theorem]{Conclusion}
\newtheorem{condition}[theorem]{Condition}
\newtheorem{conjecture}[theorem]{Conjecture}
\newtheorem{corollary}[theorem]{Corollary}
\newtheorem{criterion}[theorem]{Criterion}
\newtheorem{definition}{Definition}
\newtheorem{example}[theorem]{Example}
\newtheorem{exercise}[theorem]{Exercise}
\newtheorem{lemma}{Lemma}
\newtheorem{notation}[theorem]{Notation}
\newtheorem{problem}[theorem]{Problem}
\newtheorem{proposition}{Proposition}
\newtheorem{remark}[theorem]{Remark}
\newtheorem{solution}[theorem]{Solution}
\newtheorem{summary}[theorem]{Summary}


\newcommand{\dofigure}[2]{\begin{figure}[h!]
\center %$#1$
\includegraphics[width=0.75\linewidth]{exhibits/#1}
\caption{#2\label{fig:#1}}
\end{figure}}

\newcommand{\dotable}[2]{\begin{table}[h!]
\caption{#2\label{tab:#1}}
\center %$#1$
\includegraphics[width=0.8\linewidth]{exhibits/#1}
\end{table}}

\newcommand{\Var}{\mathbf{Var}}
\newcommand{\Cov}{\mathbf{Cov}}

\begin{document}

\title{Bridges\thanks{We thank Treb Allen, Costas Arkolakis, Jerry Carlino, Arnaud Costinot, Jonathan Eaton, Gene Grossman, Cynthia Kinnan, Eduardo Morales, Stephen Redding, Esteban Rossi-Hansberg, Marko Terviö and Jonathan Vogel and seminar audiences at Aalto University, Brown University, ECARES and UC Louvain for useful comments. B\'{a}lint Menyh\'{e}rt, Zolt\'an S\'api and Vukan Vujic provided excellent research assistance. Some of the research was carried out while Armenter and Koren were visiting the William R. Rhodes Center at Brown University, and we are grateful for their hospitality. Koren acknowledges funding from European Research Council Starting Grant No. 313164. The paper does not represent the views of the Federal Reserve Bank of Philadelphia or the Federal Reserve System.}}
\author{Roc Armenter, Mikl\'{o}s Koren and D\'{a}vid Kriszti\'{a}n Nagy%
\thanks{%
Armenter: Federal Reserve Bank of Philadelphia. Koren: Central European
University, MTA KRTK and CEPR. Nagy: Princeton University.}}
\maketitle

\begin{abstract}
We build a continuous-space theory of trade in which people in a region
agglomerate to exploit trading opportunities with another region. The
regions are separated by a river, which can be crossed anywhere, but more
cheaply at bridges. In the model, most trade takes place via bridges,
leading to a key prediction that population density declines with distance
to the bridge. We derive additional predictions about the spatial
distribution of population and test them on current high-resolution population
density data around twelve major American rivers. The data is mostly
consistent with our model. 
In a historical event study of 19th-century bridges on these rivers, we find that the neighborhood of bridges developed faster after the bridge was built. Also, the two sides of the bridge converged in development, highlighting the connecting role of the bridge.
More generally, our results suggest that
economies of density arising from transport infrastructure can help explain
why and where people agglomerate.
\end{abstract}
\newpage
\section{Introduction}

People agglomerate in cities to minimize broadly defined trade costs. Shorter distances facilitate the trade of goods and services, the exchange of ideas, and maximize other spatial externalities. All of these mechanisms motivate people to live closer to \emph{one another}, but do not tell us \emph{where} cities emerge in space.

Moreover, as we later show, geographic proximity to others varies only smoothly in space, so it fails to capture the \emph{magnitude} of agglomeration, which is very uneven in space. The ten most densely populated ZIP codes in the U.S. have an average population density of XX. Just 10 miles away from these ZIP codes, the average population density is a fraction, XX.

We argue that bridges and, more broadly, natural or man-made crossing points help explain both the location and the magnitude of spatial agglomeration. First, a bridge provides a focal point to which people want to stay close. This will also be true for other pointlike amenities that people or businesses value, such as a spring, a mine or a marketplace. Second, the very function of bridges is to connect people with others. New Jersey residents will value a bridge on the Delaware more if there are more people on the Pennsylvania side, and vice versa. This interaction magnifies the forces of agglomeration around the bridge.

%% if this is our motivation, we will need counterfactuals in the model to discuss how much bridges contribute to agglomeration

To understand how bridges contribute to agglomeration, we build a continuous-space model of trade and economic geography. In our model, people choose their location on a homogeneous plane divided by
a single linear feature (a \textquotedblleft river\textquotedblright ). Given XX, people want to trade with one another, including others across the river. Trading, however, is costly. The cost
increases in the distance travelled, and crossing the river entails
additional costs. The river can be crossed in two ways: by boat at any
point, and on existing bridges for a lower cost. For a given set of bridges,
we study the patterns of specialization and the distribution of population
(and economic activity) in space.

The model explains agglomeration through three steps. First, bridges reduce trade costs, thereby increasing the attractiveness of riverside locations. Second, because they only do so at given points on the river, they introduce additional spatial variation in trade costs. This ``warping of space'' implies that bridge locations become even more important relative to other riverside locations. Third, people move to low-trade-cost locations, making these locations even more attractive to other traders. As we later explain, this third channel is especially strong for crossing points.

Our work is motivated by the historical relevance of crossing points, often
referenced in city names, such as Ox\emph{ford} and Cam\emph{bridge}.
Although the emergence of such crossing points is not exogenous, they may
lead to further and faster economic development and agglomeration. Writing
about the Upper Black Eddy--Milford bridge on the Delaware, built in 1842,
Dale (2003, p. 43) concludes that ``[t]his new crossing brought additional
business to this part of the river valley. It gave farmers and small
industrualists in the area quick access to the Delaware Canal in
Pennsylvania. And this increased use brought additional funds in the form of
dividends to the stockholders of the Upper Black Eddy-Milford Bridge. By
1844, business in the now growing town of Milford included three stores,
three taverns, twelve to fiteen mechanics' shops, a flour mill, and two new
sawmills that made lumber trade, here, an especially important business. The
town also had many non-commercial structures, including forty-five homes,
two churches, and a fine school. Upper Black Eddy on the Pennsylvania side
of the river directly opposite Milford was a favorite stop for timber
raftsmen in the early days. By the mid-nineteenth century the bridge brought
even more business. Upper Black Eddy was booming, too. It had forty houses,
three hotels, and several stores and shops.''

To evaluate the model more systematically, we test its predictions on twelve
major North American rivers: the Arkansas, the Colorado, the Columbia, the Connecticut, the Delaware, the Hudson, the Mississippi, the Missouri, the Ohio, the Red River, the Snake River and the Tennessee. In doing so, we rely on
high-resolution population density data, the precise path of rivers and the
locations of bridges.

First, we estimate how population density varies with distance to the river
and distance to the nearest bridge. It declines with distance to all twelve 
major rivers except for the Colorado and the Delaware. Except for the 
Hudson, population density declines with distance to the nearest bridge.

Second, we check whether population is more clustered at the river. We
calculate the coefficient of variation of population density within 10 miles
of the river, and find that it is higher than between 20 and 30 miles from
the river. That is, there is more variation in population along the river
than inland, as predicted by the model.

%% here we should test space warping. population gradients, clustering.

Third, we measure the correlation of population densities between the left
and the right bank with and without taking bridges into account. For all twelve
rivers, the correlation between the two banks is strongly positive (with an
average of 0.47), suggesting that people agglomerate near the same points on
either side of the river. When looking at bridges only, however, we find
that correlations are substantially lower (average 0.39) between the two
sides of the bridge. This is consistent with the model, where population
density is a decreasing function of trade costs. Moving away from a bridge
along the river, trade costs increase both on the left and on the right bank
of the river, leading to a comovement in population density across the two
banks. Therefore, starting from a bridge, the longer the interval over which
we calculate the correlation coefficient, the larger value we find.

To study whether bridges lead to development or vice versa, we conduct an event study, using historical data on 19th century bridges. To proxy for the emergence of townships, we use data on post offices. We find that more post offices appeared after the bridge was built, relative to similar non-bridge locations. Bridges explain about a 10 percent growth in the number of post offices in 30 years. We also find that the two sides of the bridge become more similar after the bridge is built.

\bigskip

Our paper is related to two strands of the literature. First, it is related
to an increasing number of papers which model space as ordered and
continuous -- a much more realistic assumption than the ones used in
classical economic geography models.\ Rossi-Hansberg (2005), Desmet and
Rossi-Hansberg (2014), and Co\c{s}ar and Fajgelbaum\ (2013) characterize the
spatial distribution of economic activity over a line segment in Ricardian
models with agglomeration externalities. Fabinger (2011) and Allen and
Arkolakis (2014), on the other hand, examine the implications of
neoclassical models with CES\ preferences on the geographic distribution of
economic activity.\ All of these papers come to the conclusion that \textit{%
lower-dimensional trade barriers and trade infrastructure} -- ports in Co%
\c{s}ar and Fajgelbaum (2013), borders in Rossi-Hansberg (2005) and Fabinger
(2011), and highways in Allen and Arkolakis (2014) -- might
have a significant impact on how population, income, and other relevant
economic variables are distributed in space. In most of the cases, trade
infrastructure has an agglomeration-creating effect since people want to
exploit spatial proximity to trading opportunities, while trade barriers
repel agglomeration in these models.\footnote{%
Rossi-Hansberg (2005), however, points to a case in which more trade
restrictions on the border are responsible for creating agglomeration.} Our
main contribution to this literature is that, to the best of our knowledge,
we are the first to study the role of bridges, or other point-like transport
infrastructure, in creating agglomeration.

The second literature related to this paper studies the role of transport
infrastructure in development in more empirical settings. The most closely related 
paper is Tompsett (2013), who shows that bridge construction over the Mississippi 
and the Ohio contributed substantially to the growth of counties close to newly 
built bridges. Donaldson (2012) and Donaldson and Hornbeck (2013) study the expansion
of railroads in the 19th century, and come to the conclusion that it was a crucial determinant of local
development in India and the U.S., respectively.\ Baum-Snow et al. (2012)
and Duranton et al. (2012), on the other hand, find that highways have been
playing an important role in city development in China and the U.S. The fact
that these effects are likely to be long-lasting is pointed out by Bleakley
and Lin (2012), who find that pre-19th-century portage sites remain
population centers, despite the fact that their advantage in transportation
have been obsolete for long. Relative to this second strand of the
literature, we identify a new mechanism for agglomeration. In these papers,
transport infrastructure is assumed to reduce trade costs, but is not a
source of agglomeration itself. In our model, bridges not only make trade
between two regions cheaper, but also serve as focal points of agglomeration.

\bigskip 

The structure of the paper is as follows. Section 2 describes the model
together with the set of predictions that the model provides, while Section
3 presents the data, the empirical strategy, and the cross-sectional results. 
Section 4 discusses the historical event study conducted. Section 5 concludes.

\section{A model of trading across a river}

We model production and trade in continuous space. There are a continuum of
workers freely choosing location on a plane that is separated by a river.
They produce two goods, using land and their labor. The two sides of the
river differ in relative productivities, leading to Ricardian gains from
trade across the river. There are no gains from within-region trade. (Most
of our results survive if all agents specialize fully, such as in Allen and
Arkolakis, 2014.) Transportation is costly, giving incentives to move close
to trade opportunities.

We study how the relative price of the two goods varies in space, and the
patterns of specialization and agglomeration. For a fixed set of bridges, we
derive a handful of predictions on the equilibrium distribution of
population around the river and bridges.

\subsection{Geography}

We concentrate on a compact and connected subset $S$ of
the sphere that represents the globe. A circle segment called the \textit{%
river} divides $S$ into two parts: a region called Home ($H$), and a region called 
Foreign ($F$) -- see Figure \ref{fig:figurem1}.\footnote{%
A\ circle on the sphere is equivalent to a straight line on the plane:\ it
is the shortest path between any two points that lie on it. Also note that
the river is assumed to have zero width. However, this assumption is without
loss of generality because the only relevant geographical feature of the
river in our model is the cost of crossing it, which is given exogenously.}
Locations (i.e., points in $S$) are indexed by the triplet $\left( R,\ell
,h\right) $, where $R\in \left \{ H,F\right \} $ is the region to which the
location belongs, $\ell $ is the distance of the location from the river,
and $h$ is the distance of the location from an arbitrarily chosen circle $%
h=0$ that is perpendicular to the river. (In other words, $h$ represents the river mile.) For simplicity, we refer to $\ell $
as \textquotedblleft longitude,\textquotedblright \ and to $h$ as
\textquotedblleft latitude.\textquotedblright \ Finally, there is a finite
set of latitudes $h_{1},\ldots ,h_{B}$ at which \textit{bridges} span the
river.

\dofigure{figurem1}{Geography of the river}

There are two goods, denoted by $X$ and $M$.\ Shipment of goods is costly.
Land shipping of good $i$ involves an iceberg cost of $e^{t_{i}d}$, where $%
t_{i}$ is a positive constant, and $d$ is total distance traveled. Crossing
the river entails additional costs. The river can be crossed in two ways:\
(1) by boat at any point, at an iceberg cost of $e^{\tau _{i}^{0}}$, or\ (2)
through bridge $b\in \left \{ 1,\ldots ,B\right \} $, at an iceberg cost of $%
e^{\tau _{i}^{b}}$, where $\tau _{i}^{0},\tau _{i}^{b}>0$, and the value of $%
\tau _{i}^{b}$ can potentially vary with $b$.

Finally, the spatial distribution of factor endowments is as follows.\ There
is a mass of $N$ workers in $S$, each of them supplying one unit
of labor inelastically. Workers are freely mobile across space.\ Also, each location $\left(
R,\ell ,h\right) $ is endowed with a strictly positive amount of land $%
\lambda \left( R,\ell ,h\right) $. Land is owned by local landlords.

\subsection{Consumption}

Workers have Cobb--Douglas preferences over goods $X$ and $M$, spending half
of their income on each good. Therefore, the representative worker at
location $\left( R,\ell ,h\right) $ has indirect utility%
\begin{equation}
u\left( R,\ell ,h\right) =\frac{w\left( R,\ell ,h\right) }{P_{X}\left(
R,\ell ,h\right) ^{\frac{1}{2}}P_{M}\left( R,\ell ,h\right) ^{\frac{1}{2}}},
\end{equation}%
where $w\left( R,\ell ,h\right) $ is the wage at $\left( R,\ell ,h\right) $,
and $P_{X}\left( R,\ell ,h\right) $ and $P_{M}\left( R,\ell ,h\right) $ are
the local prices of the $X$- and $M$-goods, respectively. Workers move to 
the location at which their indirect utility is largest.

Landlords have the same preferences as workers. Landlords are immobile, and
do not work. We assume that the number of landlords is small enough that we
can approximate total population by the number of workers at each location.

\subsection{Production}

Both goods are produced under a Cobb--Douglas technology and constant returns, using labor and
land. Using $n_{i} \left( R, \ell, h \right)$ units of labor \textit{per unit of land}, 
a firm operating in the $i$-sector at $\left( R, \ell, h \right)$ can produce
\begin{equation*}
q_{i} \left( R, \ell, h \right) = \frac{n \left( R, \ell, h \right)^{\gamma}}{a_{i}^{R}} 
n_{i} \left( R, \ell, h \right)^{\alpha}
\end{equation*}
units of output per unit of land, where $0 < \alpha < 1$ and $0 \leq \gamma \leq 1-\alpha$ are constants. Total factor productivity
$\frac{n \left( R, \ell, h \right)^{\gamma}}{a_{i}^{R}}$ depends on (1) population 
per unit of land $n \left( R, \ell, h \right)$ (agglomeration externalities), and (2) 
a region- and sector-specific term $a_{i}^{R}$, which represents differences in 
comparative advantage across regions. Note that the
autarky price of the $X$-good relative to the $M$-good is $%
p_{A}^{R}=\frac{a_{X}^{R}}{a_{M}^{R}}$ at each location in region $R$.

Both sectors are characterized by perfect competition
at each location. Therefore, a firm which operates in the $i$-sector at $%
\left( R,\ell ,h\right) $ solves the problem%
\begin{equation*}
\max_{n_{i}\left( R,\ell ,h\right) }P_{i}\left( R,\ell ,h\right) \frac{%
n \left( R,\ell ,h\right) ^{\gamma }}{a_{i}^{R}} n_{i}\left( R,\ell ,h\right) ^{\alpha } 
- w\left( R,\ell ,h\right)
n_{i}\left( R,\ell ,h\right) -r\left( R,\ell ,h\right) ,
\end{equation*}%
where $r\left( C,\ell ,h\right) $ is rent per unit unit of land.

The first-order condition to the firm's\ maximization problem implies%
\begin{equation}
n_{i}\left( R,\ell ,h\right) =\alpha ^{\frac{1}{1-\alpha }} n \left( R,\ell ,h\right)^{\frac{\gamma}{1-\alpha}} 
\left(a_{i}^{R}\right) ^{-\frac{1}{1-\alpha }}\left[ \frac{P_{i}\left( R,\ell
,h\right) }{w\left( R,\ell ,h\right) }\right] ^{\frac{1}{1-\alpha }}.
\end{equation}%
Hence, if a good is produced at location $\left( R,\ell ,h\right) $, then
the mass of workers in the good's production is positively linked to the
good's local price relative to the nominal wage.

\subsection{Equilibrium}

Now we define a competitive equilibrium in this economy.

\begin{definition}
An equilibrium is a set of functions $P_{X}$, $P_{M}$, $n_{X}$, $n_{M}$, $n$%
, $\lambda _{X}$, $\lambda _{M}$, $w$ and $r$, as well as a utility level $u$ such that

(1) utility of workers is maximized and equalized across locations:%
\begin{equation*}
\frac{w\left( R,\ell ,h\right) }{P_{X}\left( R,\ell ,h\right) ^{\frac{1}{2}%
}P_{M}\left( R,\ell ,h\right) ^{\frac{1}{2}}}=u
\end{equation*}%
for all $R\in \left \{ H,F\right \} $, $\ell $, and $h$,

(2) profits are maximized and driven down to zero:%
\begin{equation*}
P_{i}\left( R,\ell ,h\right) \frac{%
n \left( R,\ell ,h\right) ^{\gamma }}{a_{i}^{R}} n_{i}\left( R,\ell ,h\right) ^{\alpha } 
- w\left( R,\ell ,h\right)
n_{i}\left( R,\ell ,h\right) -r\left( R,\ell ,h\right) = 0
\end{equation*}
for all $R\in \left \{ H,F\right \} $, $\ell $, and $h$,

(3) local land markets clear:%
\begin{equation*}
\lambda _{X}\left( R,\ell ,h\right) +\lambda _{M}\left( R,\ell ,h\right)
=\lambda \left( R,\ell ,h\right)
\end{equation*}%
for all $R\in \left \{ H,F\right \} $, $\ell $, and $h$, where $\lambda
_{i}\left( R,\ell ,h\right) $ denotes local land usage by sector $i$,

(4) local and global labor markets clear:%
\begin{eqnarray*}
\frac{\lambda _{X}\left( R,\ell ,h\right) n_{X}\left( R,\ell ,h\right)
+\lambda _{M}\left( R,\ell ,h\right) n_{M}\left( R,\ell ,h\right) }{\lambda
\left( R,\ell ,h\right) } &=&n\left( R,\ell ,h\right) \\
\int_{H}\lambda \left( H,\ell ,h\right) n\left( H,\ell ,h\right) d \ell dh + 
\int_{F}\lambda \left( F,\ell ,h\right) n\left( F,\ell ,h\right) d \ell dh &=&N
\end{eqnarray*}%
for all $R\in \left \{ H,F\right \} $, $\ell $, and $h$,

(5)\ there is no arbitrage possibility within regions:%
\begin{equation*}
P_{i}\left( R,\ell ,h\right) \leq e^{t_{i}d\left[ \left( R,\ell ,h\right)
,\left( R,\ell ^{\prime },h^{\prime }\right) \right] }P_{i}\left( R,\ell
^{\prime },h^{\prime }\right)
\end{equation*}%
for all $\left( R,\ell ,h\right) $ and $\left( R,\ell ^{\prime },h^{\prime
}\right) $, where $d\left[ \left( R,\ell ,h\right) ,\left( R,\ell ^{\prime
},h^{\prime }\right) \right] $ denotes the distance between the two
locations, and we have equality if $\left( R,\ell ^{\prime },h^{\prime
}\right) $ ships good $i$ through $\left( R,\ell ,h\right) $,

(6) there is no arbitrage possibility over the river:%
\begin{equation*}
P_{i}\left( R,0,h\right) \leq e^{\tau _{i}^{0}}P_{i}\left( R^{\prime
},0,h\right)
\end{equation*}%
for all $R$, $R^{\prime }$ and $h$, and we have equality if region $%
R^{\prime }$ exports good $i$ at $h$ by boat,

(7) there is no arbitrage possibility over bridges:%
\begin{equation*}
P_{i}\left( R,0,h_{b}\right) \leq e^{\tau _{i}^{b}}P_{i}\left( R^{\prime
},0,h_{b}\right)
\end{equation*}%
for all $R$, $R^{\prime }$ and $b\in \left \{ 1,\ldots ,B\right \} $, and we
have equality if region $R^{\prime }$ exports good $i$ through bridge $b$,

(8) trade is balanced between each pair of locations.
\end{definition}

Let us introduce the notation $p\left( R,\ell ,h\right) =\frac{P_{X}\left(
R,\ell ,h\right) }{P_{M}\left( R,\ell ,h\right) }$, that is, $p\left( R,\ell
,h\right) $ is the \textit{relative price} of the $X$-good at location $%
\left( R,\ell ,h\right) $. What is the pattern of specialization in
equilibrium? By constant returns to scale, this only depends on the
relationship between the equilibrium relative price and the autarky
relative price. In particular,

\begin{itemize}
\item $\left( R,\ell ,h\right) $ is fully specialized in good $X$ if $%
p\left( R,\ell ,h\right) >p_{A}^{R}$, implying $n\left( R,\ell ,h\right)
=n_{X}\left( R,\ell ,h\right) $,

\item $\left( R,\ell ,h\right) $ is fully specialized in good $M$ if $%
p\left( R,\ell ,h\right) <p_{A}^{R}$, implying $n\left( R,\ell ,h\right)
=n_{M}\left( R,\ell ,h\right) $,

\item if $\left( R,\ell ,h\right) $ is incompletely specialized, then $%
p\left( R,\ell ,h\right) =p_{A}^{R}$, and $n\left( R,\ell ,h\right)
=n_{X}\left( R,\ell ,h\right) =n_{M}\left( R,\ell ,h\right) $ by (2).
\end{itemize}

Also note that, due to trade costs, any location that is incompletely
specialized is necessarily in autarky:\ a consumer at such a place would
never find it optimal to buy any of the two goods from another location.

We assume that Home has a comparative advantage in $X$. In other words, no Home location specializes in good $M$, and no Foreign
location specializes in good $X$. Given the trade costs, 
a sufficient condition for this is%
\begin{equation*}
p_{A}^{H}<p_{A}^{F}e^{-\max \left \{ \max_{b}\left( \tau _{X}^{b}+\tau
_{M}^{b}\right) ,\tau _{X}^{0}+\tau _{M}^{0}\right \} }.
\end{equation*}%
Also note that there can be no within-region trade in equilibrium:
locations that are in autarky do not trade at all, whereas locations
specialized in the region's export good only trade with the other region.

We then have the following proposition that is a generalization of
Proposition $1$ in Co\c{s}ar and Fajgelbaum (2013).

\begin{proposition}
In any equilibrium, each region $R$ is a union of two disjoint sets 
$T^{R}$ and $A^{R}$ such that

(i) all locations in $T^{R}$ trade with the other country,

(ii) all locations in $A^{R}$ that are not on the boundary of $T^{R}$
are in autarky, and

(iii) locations in $A^{R}$ that are on the boundary of $T^{R}$ are
indifferent between trade and autarky.

Moreover, for each region $R$ and latitude $h$, there exists a longitude $%
\widehat{\ell }\left( R,h\right) $ such that $\left( R,\ell ,h\right) \in
T^{R}$ for all $\ell <\widehat{\ell }\left( R,h\right) $, and $\left( R,\ell
,h\right) \in A^{R}$ for all $\ell >\widehat{\ell }\left( R,h\right) $.
\end{proposition}

\begin{proof}
See Appendix.
\end{proof}

Figure \ref{fig:figurem2} is a graphical illustration of Proposition $1$.\
As one can see, trading locations $T^{R}$ are closer to the
river than locations in autharky $A^{R}$ for each latitude.

\dofigure{figurem2}{Spatial specialization}

Combining equations (1) and (2), and using the equalization of utility in
equilibrium, we can relate the equilibrium spatial distribution of
population to the equilibrium spatial distribution of relative prices:%
\begin{equation}
n\left( H,\ell ,h\right) =n_{X}\left( H,\ell ,h\right) =\alpha ^{\frac{1}{%
1-\alpha-\gamma }}\left( u a_{X}^{H}\right) ^{-\frac{1}{1-\alpha-\gamma }}p\left(
H,\ell ,h\right) ^{\frac{1}{2\left( 1-\alpha-\gamma \right) }}
\end{equation}%
and%
\begin{equation}
n\left( F,\ell ,h\right) =n_{M}\left( F,\ell ,h\right) =\alpha ^{\frac{1}{%
1-\alpha-\gamma }}\left( u a_{M}^{F}\right) ^{-\frac{1}{1-\alpha-\gamma }}p\left(
F,\ell ,h\right) ^{-\frac{1}{2\left( 1-\alpha-\gamma \right) }}.
\end{equation}

That is, within-region differences in population density are solely
driven by differences in the relative price.\ At Home, locations that have a
high $p$ offer a high price of the export good and a low price of the import
good. Hence, many people decide to move to these locations.\ On the
contrary, a location with a high $p$ is not attractive in the Foreign
region; thus, such locations are characterized by low population density in
equilibrium.

Using equations (3) and (4), the model generates two predictions on the
distribution of population, summarized in Propositions $2$ and $3$.\footnote{%
The Appendix contains the proofs of these propositions.}

\begin{proposition}[Concentration at the river]
Take a region $R$, and restrict attention to a ``rectangular'' subset of
locations $\left \{ \left( R,\ell ,h\right) :\underline{\ell }\leq \ell \leq 
\overline{\ell },\underline{h}\leq h\leq \overline{h}\right \} \subset R$.
Then average population density of locations at distance $\ell $ from the
river is at least as high as average population density of locations at
distance $\ell ^{\prime }>\ell $ from the river.
\end{proposition}

\begin{proposition}[Concentration at bridges]
In any region $R$, take two locations $\left( R,\ell ,h\right) $ and $%
\left( R,\ell ^{\prime },h^{\prime }\right) $ which trade over the same
bridge. Then $n\left( R,\ell ,h\right) >n\left( R,\ell ^{\prime },h^{\prime
}\right) $ if and only if $\left( R,\ell ,h\right) $ is closer to the bridge
than $\left( R,\ell ^{\prime },h^{\prime }\right) $. As a consequence,
locations at bridges over which trade takes place coincide with the local maxima
of $n\left( R,\ell ,h\right) $ if boat trade is prohibitively costly.
\end{proposition}

Proposition $2$ points to the fact that riverside locations are always more 
attractive than locations far away from the river since they have lower trade 
costs. In the absence of bridges, this is the only source of heterogeneity in trade costs 
and, hence, in population density, across locations. As a consequence, level 
curves of population density are parallel to the river without bridges. The real-world 
analog of this would be a large number of small villages doing boat trade 
along the river. Proposition $3$, on the other hand, shows that bridges introduce 
an additional source of heterogeneity in trade costs, and hence in the 
concentration of population (``warping of space''). In the extreme case of no 
boat trade, level curves of population density become ``circles'' around bridges. 
The real-world analog of this would be a small set of high-density population clusters 
located at bridges, which might call \textit{cities}.

Inspecting equations (3) and (4), one can immediately notice that the above 
relationships between distance from the river, or bridges, and population density 
are stronger in the presence of agglomeration externalities ($\gamma > 0$) than 
without such externalities ($\gamma = 0$). The higher the value of $\gamma$, the 
faster population density falls with relative prices, thus with distance. One can 
therefore conclude that agglomeration externalities bring about an additional 
incentive to co-locate, which makes concentration at the river and bridges even stronger.

\subsection{Random variation in productivity}

This section presents a generalization of the model in which productivity is
not necessarily evenly distributed within countries. We do this to account for idiosyncratic variation in population density in equilibrium.

Let $a_{i}\left( R,\ell
,h\right) $ be the unit cost of production in sector $i\in \left \{
X,M\right \} $ at location $\left( R,\ell ,h\right) $.\ We assume that $%
a_{i}\left( R,\ell ,h\right) $ are random variables, each with marginal cdf $%
G_{i}^{R}\left( \cdot \right) $, but not necessarily independent. That is,
our specification allows for both spatial and cross-industry correlation of
productivity draws.\ Finally, we assume that $G_{M}^{H}\left( \cdot \right) $
and $G_{X}^{F}\left( \cdot \right) $ are such that Home locations specialize
in the $X$-good, and Foreign locations specialize in the $M$-good with
probability one.\footnote{%
The easiest way to satisfy this restriction is to assume that $a_{M}\left(
H,\ell ,h\right) $, or $a_{X}\left( F,\ell ,h\right) $, or both, are ``very
large'' with probability one.}

Under these assumptions, equations\ (3) and (4) can be written as%
\begin{equation}
n\left( H,\ell ,h\right) =n_{X}\left( H,\ell ,h\right) =\alpha ^{\frac{1}{%
1-\alpha-\gamma }} u^{-\frac{1}{1-\alpha-\gamma }}a_{X}\left( H,\ell
,h\right) ^{-\frac{1}{1-\alpha-\gamma }}p\left( H,\ell ,h\right) ^{\frac{1}{2\left(
1-\alpha-\gamma \right) }}  \tag{3'}
\end{equation}%
and%
\begin{equation}
n\left( F,\ell ,h\right) =n_{M}\left( F,\ell ,h\right) =\alpha ^{\frac{1}{%
1-\alpha-\gamma }}u^{-\frac{1}{1-\alpha-\gamma }}a_{M}\left( F,\ell
,h\right) ^{-\frac{1}{1-\alpha-\gamma }}p\left( F,\ell ,h\right) ^{-\frac{1}{%
2\left( 1-\alpha-\gamma \right) }}  \tag{4'}
\end{equation}

Equations (3') and (4') imply that Propositions $2$ and $3$ still hold in
expectation, i.e., if one replaces \textquotedblleft population
density\textquotedblright \ at a given location, $n\left( R,\ell ,h\right) $%
, by \textquotedblleft expected population density\textquotedblright \ at
the location, $\mathbf{E}n\left( R,\ell ,h\right) $.

Generalizing the distribution of productivity comes at the expense of more restrictions on geographical structure. First, we assume that the
two regions are mirror images of each other, that is, (1) $\left( H,\ell
,h\right) \in H$ if and only if $\left( F,\ell ,h\right) \in F$ for all $%
\ell $ and $h$, and (2) the distribution of land is such that $\lambda
\left( H,\ell ,h\right) =\lambda \left( F,\ell ,h\right) $ for all $\ell $
and $h$. Second, we assume that the productivity of the good in which the region
specializes (good $X$ at Home, and good $M$ in Foreign) is drawn from the
same distribution in the two countries, that is, $G_{X}^{H}\left( \cdot
\right) =G_{M}^{F}\left( \cdot \right) $. Third, we assume that trade over
every bridge has the same iceberg cost:\ $\tau _{i}^{b}=\tau _{i}$, and boat
trade is prohibitively costly.

Under these assumptions, one can show that the distribution of relative
prices along the river takes the form%
\begin{eqnarray*}
p\left( H,0,h\right) &=&e^{2p-\left( t_{X}+t_{M}\right) \delta \left(
h\right) } \\
p\left( F,0,h\right) &=&e^{2p+\tau _{X}+\tau _{M}+\left( t_{X}+t_{M}\right)
\delta \left( h\right) },
\end{eqnarray*}%
where $p$ is a constant, and $\delta \left( h\right) $ denotes the distance
of location $\left( 0,h\right) $ from the closest bridge. Denote $\tau =%
\frac{\tau _{X}+\tau _{M}}{2}$\ and $t=\frac{t_{X}+t_{M}}{2}$. Then (3') and
(4') yield, in logs,%
\begin{eqnarray*}
\log n\left( H,0,h\right) &=&\frac{1}{1-\alpha-\gamma }\left[ \log \alpha -\log
u+p-\log a_{X}\left( H,0,h\right) -t\delta \left( h\right) \right] \\
\log n\left( F,0,h\right) &=&\frac{1}{1-\alpha-\gamma }\left[ \log \alpha -\log
u-p-\tau -\log a_{M}\left( F,0,h\right) -t\delta \left( h\right) \right]
\end{eqnarray*}

Therefore,%
\begin{equation*}
\Cov\left[ \log n\left( H,0,h\right) ,\log n\left( F,0,h\right) \right] =%
\frac{1}{\left( 1-\alpha-\gamma \right) ^{2}}\left[ C_{LR}+t^{2}\Var\left[ \delta
\left( h\right) \right] \right]
\end{equation*}%
where $C_{LR}=\Cov\left[ \log a_{X}\left( H,0,h\right) ,\log a_{M}\left(
F,0,h\right) \right] $ is the covariance between productivity realizations
of the two banks of the river. Now since%
\begin{equation*}
\Var\left[ \log n\left( H,0,h\right) \right] =\frac{1}{\left( 1-\alpha-\gamma
\right) ^{2}}\left[ \sigma ^{2}+t^{2}\Var\left( \delta \left( h\right)
\right) \right] =\Var\left[ \log n\left( F,0,h\right) \right]
\end{equation*}%
where $\sigma ^{2}$ is the common variance of $\log a_{X}\left( H,0,h\right) 
$ and $\log a_{M}\left( F,0,h\right) $, we obtain that the correlation
between left- and right-bank log population density is%
\begin{equation}
\rho =\frac{C_{LR}+t^{2}\Var\left[ \delta \left( h\right) \right] }{\sigma
^{2}+t^{2}\Var\left[ \delta \left( h\right) \right] }=1-\frac{\sigma
^{2}-C_{LR}}{\sigma ^{2}+t^{2}\Var\left[ \delta \left( h\right) \right] }.
\end{equation}

This equation allows us to provide the following two predictions on the
distribution of population along the river.

\begin{proposition}[Left- and right-bank density positively correlated]
If left- and right-bank log productivities are positively correlated or
uncorrelated, then the correlation between left- and right-bank log
population density is positive.
\end{proposition}

\begin{proof}
If log productivities are positively correlated or uncorrelated, then $%
C_{LR}\geq 0$. Then equation (5)\ immediately implies%
\[
\rho \geq 1-\frac{\sigma ^{2}}{\sigma ^{2}+t^{2}\Var\left[ \delta \left(
h\right) \right] }>0\text{.}
\]
\end{proof}

\begin{proposition}[Lower correlation at bridges]
The correlation between left- and right-bank population density is lower at
(trading) bridges than in general.
\end{proposition}

\begin{proof}
Calculating the correlation coefficient at trading bridges only, we find%
\[
\rho _{bridges}=1-\frac{\sigma ^{2}-C_{LR}}{\sigma ^{2}}=\frac{C_{LR}}{%
\sigma ^{2}}
\]%
because $\delta \left( h\right) \equiv 0$, hence $\Var\left[ \delta \left(
h\right) \right] =0$ in this case. $\rho _{bridges}<\rho $ follows from
comparing this to equation (5).
\end{proof}

The intuition for Proposition 5 is as follows.\ As we move away from a
bridge along the river, trade costs increase by as much on the left bank as
on the right bank of the river. This leads to a comovement in the terms of
trade ($p$ in Home, and $p^{-1}$ in Foreign) on the two banks, and
hence to a comovement in\ Home and Foreign population density (which are
increasing power functions of the terms of trade). This comovement in
densities acts against the variation caused by fluctuations in productivity.
Thus, starting from a bridge, the longer the interval over which we
calculate the correlation coefficient, the larger value we find for $\rho $.

Note that the existence of bridges over which trade takes place is crucial
for this result: the above mentioned comovement in trade costs is absent
whenever people trade by boat, or are in autarky. Hence, the fact that this
prediction is verified in the data can be taken as a clear indication that
bridges matter for the spatial distribution of economic activity.

\section{Rivers and population density}

We test the model's predictions on twelve major rivers of the continental U.S.: the Arkansas, the Colorado, the Columbia, the Connecticut, the Delaware, the Hudson, the Mississippi, the Missouri, the Ohio, the Red River, the Snake River and the Tennessee.

\subsection{Mapping model to data}

In the model, each location on either side of the river is characterized by two coordinates: its
distance from the river (longitude) and its distance along the river from a
chosen rivermile (latitude). In
reality, rivers are not straight lines. To calculate these two relevant
coordinates, we proceed as follows.

Let river $\varsigma:\mathbb{R}\to \mathbb{R}^2$ be a parametric
curve mapping rivermiles into points on the plane. $\varsigma(0)$ is the
vector of geocoordinates of the river's mouth, $\varsigma(1)$ is the geocoordinate
of the first rivermile, etc. For any point $(x,y)$, we can determine the
river-coordinates as follows:
\begin{align*}
\ell(x,y,\varsigma) &\equiv \min_m d[(x,y),\varsigma(m)], \\
h(x,y,\varsigma) &\equiv \arg \min_m d[(x,y),\varsigma(m)],
\end{align*}
where $d:\mathbb{R}^4\to \mathbb{R}^+$ measures the distance between a pair
of points.

That is, distance is measured as distance to the nearest point of the river,
and $h$ is measured as the rivermile of this neaerst point. For straight
rivers, these measures exactly correspond to the ones used in the model.

Note that the $(x,y)\rightarrow (\ell ,h)$ mapping is not a bijection. While
there is only one nearest point with probability one, there may be multiple $%
(x,y)$ points on the plane for which $m$ is the closest rivermile.

The use of this mapping is illustrated in Figure \ref%
{fig:delaware_leftrightdensity}, which plots population density on the left
and right bank of the Delaware as a function of rivermiles. The high-density
areas of Philadelphia (mostly right bank) and Trenton (mostly left bank) are
clearly visible.

\dofigure{delaware_leftrightdensity}{Bridges and population density on the two banks of the Delaware}

\subsection{Data}

We merge three datasets to conduct the analysis: one on the location of rivers, one on the location of bridges, and one on the current distribution of population.

We take the geocoordinates of the twelve major rivers from the ESRI Map of 
U.S.~Major Waters, containing polygons of 29,167 water surfaces, including 
rivers and lakes. After making the necessary topological corrections 
(connecting segments of the river and intermittent lakes), we determine the
centerline, the left and the right bank of each river. For the Delaware, we exclude Philadelphia
and for the Hudson, we exclude New York City from the analysis. Our results
are stronger with these cities included.

The geocoordinates of bridges, including the year of building and decommissioning (if any), come from 
\textit{bridgehunter.com}, cross-referenced with Wikipedia, the National Bridge Inventory, and historical sources. For the Hudson, we include the tunnels at Manhattan.

To measure current population density, we use Version 1 of the Global Rural-Urban
Mapping Project population density grid. This dataset provides population
count (and density) estimates for each 30 arc-second by 30 arc-second
grid cell of the U.S. (The area of these gridcells is around 0.25 km${}^2$.) We use the values from year 1995.

%\dofigure{rivermap}{Map of the twelve rivers, their bridges, and population density}

\subsection{Testing the five predictions}

\dotable{fact1}{Population density and distance to the river}

Table \ref{tab:fact1} shows how population density varies with distance to
the twelve major rivers. We measure population density within 10-mile bands
along the river. On ten out of the twelve rivers, population density between
20 and 30 miles from the river is strictly lower than within 10 miles from the river. 
This is consistent with the model, where trading opportunities on the
other side of the river lead to a density gradient.

The exceptions are the Colorado and the Delaware, where population density 
does not show a declining pattern. The Colorado river flows across areas 
with large heterogeneity in natural endowments (such as mountains, canyons, 
and deserts), which might lead to differences in population density which are 
largely unrelated to the path of the river. In the case of the Delaware, its 20 
to 30-mile neighborhood is affected by the metropolitan area of New York City.

\dotable{fact2}{Higher concentration near bridges}

The first column of Table \ref{tab:fact2} shows the correlation coefficient of log population
density along the river with distance to the nearest bridge. With the
exception of the Hudson, all rivers display very strong negative
correlation. In the model, as bridges are the focal points of agglomeration,
population density falls with distance, consistently with the facts.

Figures \ref{fig:delaware_kernel} through \ref{fig:tennessee_kernel} (in the Appendix) plot the distribution of population density near and away from bridges. For each river, the red line plots the kernel density of log population densities at gridcells that have a bridge within 3 miles. (We calculated average population density between 0 and 10 miles from the river.) The blue line plots the kernel estimate of log population densities for gridcells more than 3 miles from a bridge.

For all rivers except the Hudson, the distribution of population densities near bridges is shifted to the right. That is, average population density is higher within 3 miles of the bridge than outside this distance. This is consistent with the correlations reported in Table \ref{tab:fact2}.

We also see that there is a large variation in population densities both near and away from bridges. In particular, some locations without a bridge are as densely populated as some of those with one. This suggests that building bridges involves nontrivial costs, and not every community can overcome these costs, severely limiting their access to the other side of the river.

Table \ref{tab:fact2} also displays a measure of clustering at various distances
to the river. In particular, we calculate the coefficient of variation of population
density. This measure is high when population density varies a lot, between,
say, a large a city and sparse surroundings. It is low when many small
cities or towns are roughly evenly distributed in space. Note that the
coefficient of variation is unaffected by the overall mean population
density, which we have reported in Table \ref{tab:fact1}.

As we move farther away from the river, the coefficient of variation tends
to fall. This is in line with our theory, where the agglomerating force of
bridges can only be felt close to the river, and not farther from where
multiple bridges are equally easily accessible.

\dotable{fact3}{Correlation between two banks of the river}

Table \ref{tab:fact3} reports the correlation of population density between
the left and the right bank of the river. The model predicts that population
is going to cluster on both sides of the bridge, leading to positive
correlation across the two banks. On all twelve rivers, the correlation is
highly positive, with an average of 0.47. Part of this correlation is driven
by the mere presence of bridges. Bridges are surrounded by people on either
side of the river, whereas areas far from bridges tend to be more sparse.
Table \ref{tab:fact3} also measures this correlation at the bridges. More
specifically, we ask how population on the two sides of the bridge is
correlated. As predicted by the model, these correlations are positive, but
smaller than the unconditional correlations, with Snake River being the only exception.\footnote{%
The results are very similar if we use log population density when
calculating correlations.}

\section{A historical event study}
The cross-sectional patterns reported in the previous section are consistent with the model. However, they do not address a key identification concern. If local amenities are heterogeneous in space, than equilibrium population density will also vary in space. There will be more habitable locations on the river that are more populous. If bridges are subject to economies of scale (that is, one cannot build a smaller bridge serving half the people for half the cost), then it is these more populous places where bridges will be built.

To partially address this identification concern, we conduct an event study using historical data on early bridges. We study the neighborhood of bridge locations before and after the bridge was built. We say ``partially address'' because our model is not dynamic and so does not speak directly to the exercises reported below. We can, however, ask whether bridge regions are already more populous before the bridge is built, or grow faster only after the bridge.

We study the 19th century bridges on the twelve rivers. Table \ref{tab:river-descriptives} lists the rivers by the time of the first bridge, the total number of bridges by 1900, and their length.

\dotable{river-descriptives}{Number and date of bridges}

\subsection{Post offices and population}
To study the evolution of population before and after the bridge, we need population data in high resolution both in space and time. The decennial census is usually reported by counties, which, especially in 19th century, may encompass large areas with several townships and several bridges.

To proxy for population, we have built and use a dataset on historical post offices.\footnote{See Armenter, Koren, Menyhért, Nagy and Vujic (2014) for details.} Before the start of Rural Free Delivery in 1896, rural residents could only pick up their mail at post offices. Therefore, a post office was a crucial milestone in a township's life. 

We use the opening date of the post office as the birth date of a large enough settlement. More specifically, in any given year, we can count the number of post offices in a region to proxy for development.

Figure \ref{fig:post-office-count} reports the number of post offices over time. The number has grown steadily throughout the 19th century. 

\dofigure{post-office-count}{Number of post offices over time}

To illustrate the correlation between population and post offices, Figure \ref{fig:post-offices-1830} plots the number of offices by county against their 1830 population. There is a strong correlation between the two until a county population of about 50,000. That is, using post offices for rural counties seems like a valid proxy for population. The relationship breaks down for large counties.

\dofigure{post-offices-1830}{Population and the number of post offices across countries}


\subsection{Event study}
The units of observation are actual or potential bridge locations along each of the twelve rivers. The outcome variable is the number of post offices within 10kms of the bridge location $b$ at time $t$ on river $r$. Because this number is often zero, we estimate a Poisson model.

Figure \ref{fig:post_offices_in_bridge_buffers} shows how the outcome variable was constructed. It plots a section of Delaware river in 1830, highlighting a 10km band on either side. The bridges are surrounded by 10km discs and the post offices are denoted by dots. For each bridge, we count the post offices within the 10km disc.
%% NB: future bridges would be even better control groups
As a control group, we also include locations with no bridge. We include every 10th kilometer of the river as a control group if there is no bridge on that section.

\dofigure{post_offices_in_bridge_buffers}{Bridges and post offices around the Delaware in 1830}

The expected number of post offices depends on the time since the bridge was built. The key parameters of interest are $\beta_s$, denoting the proportional difference in the number of post offices relative to non-bridge locations.

We include river-decade dummies $\mu_{rt}$ because rivers became populated at different times. We control for the rivermile associated with a location $b$, because it may be correlated with local amenities. For example, the downstream segment of the river may be more navigable. Because rivers differ in their hydrological features, we let the effect of rivermiles vary by river.
\begin{equation}
\label{eq:poisson}
E ({n_{rbt}})
		 = \exp
		 \left[\mu_{rt}+\sum_{s=-40}^{40}\beta_s I(t=T_b+s) + \gamma_r m_{b}\right].
\end{equation}
Figure \ref{fig:pooled_event_study} plots the estimated $\beta_s$ coefficients together with their 95-percent confidence interval. Bridge locations have, on average, 40 percent more post offices than non-bridge locations even before the bridge is built. After an initial jump, the number of post offices gradually increases. Three decades after the bridge, these locations have 50 percent more post offices. That is, while there is evidence for strong pre-selection of bridge sites, we also see a marked additional development in the decades following bridge construction.\footnote{The F-test rejects that 10--30 years after the bridge the number of post offices is the same as 10--30 years before the bridge, with a $p$-value of 0.045.}

\dofigure{pooled_event_study}{Poisson estimates for post office count before and after the bridge}

\subsection{Convergence}
To test whether the development is related to the bridge, we also study the two sides of the river separately. In particular, we ask if the less populous side catches up to the more populous side after the bridge is built. While we did not prove this prediction in the model, we can argue that this is a plausible prediction. Suppose a bridge is built connecting a small and a large city. By the gravity equation, the increased trade opportunity will be more important for the small city, so it will grow faster than the large one. The size difference between the two sides will decrease.

We use a similar research design as above, but with a different outcome variable. Let $s_{rbt}$ denote the share of post offices within 10kms of location $b$ on river $r$ on the ``smaller'' side. We classify the left side as smaller if the total number of post offices on the left side of the river before the bridge is built is smaller than on the right side of the river. By construction $s_{rbt}\le 0.5$.

We ask whether the share of the smaller side converges toward 50 percent, by estimating
\[
s_{rbt}		 = 
 	\mu_{rt}+\sum_{s=-40}^{40}\beta_s I(t=T_b+s) + e_{rbt}.
\]
Again, $\beta_s$ are the treatment effects over time and we include river-decade fixed effects.

%% NB: we could do higher frequency in time
Figure \ref{fig:pooled_convergence} plots the estimated $\beta_s$s from 30 years before the bridge to 30 years after, together with their 95 percent confidence intervals. At an average bridge location, the smaller side holds about 15 percent of the post offices before the bridge is built, with 85 percent on the other side. That is, the typical bridge location is rather asymmetric.\footnote{Non-bridge locations are even more asymmetric. They feature an 8--92 percent split.} Shortly after the bridge is built, the share of the smaller side increases dramatically to 22 percent. While very far from complete convergence (which would be captured by a 50 percent share), the two sides become significantly more similar after the bridge.\footnote{The F-test rejects that 10--30 years after the bridge the number of post offices is the same as 10--30 years before the bridge, with a $p$-value of 0.002.}

\dofigure{pooled_convergence}{Share of smaller side of the river around the bridge}


\section{Conclusion}

We built a continuous-space theory of trade to explain why and where people
agglomerate. We tested the equilibrium predictions of our model on data from
twelve major American rivers, finding that spatial patterns of population
density are consistent with our model. Using historical data, we came to the 
conclusion that, although bridge placement is clearly endogenous, locations 
at which bridges were built saw more development than no-bridge locations in 
the years that followed bridge construction.

Our theory can relate to the question of whether and how trade causes
development. There are two puzzling facts about trade and development.
First, the macro correlations between trade and development (even those
using plausibly exogenous variation in trade, as Feyrer, 2009a and b) are
much larger than model-based meausures of gains from trade (Alvarez and
Lucas, 2007, Arkolakis, Costinot and Rodr\'{\i}guez-Clare, 2012). Second,
land-locked countries are much less developed than coastal countries, even
though transportation costs make up only a small fraction of broader trade
costs (Anderson and van Wincoop, 2004).

Our theory has the potential to explain these facts because trade increases
the incentives to agglomerate, which leads to external effects. These
external effects represent a multiplier of trade on development (consistent
with Fact 1). They are also stronger in coastal countries, where ports
provide a natural focal point of agglomeration (consistent with Fact 2).

In future work, we intend to estimate the agglomeration effect of bridges. The crucial identification concern is that both the location of bridges and population density are correlated with unobserved local amenities. We plan to use variation in building costs (geographical and hydrological measures) and transit traffic demand to instrument bridge location. 

\begin{thebibliography}{99}
\bibitem{} Allen, Treb and Arkolakis, Costas, 2014. ``Trade and the Topography of
the Spatial Economy.'' Quarterly Journal of Economics, forthcoming.

\bibitem{} Alvarez, Fernando and Lucas, Robert E. Jr., 2007. ``General
Equilibrium Analysis of the Eaton--Kortum Model of International Trade.''
Journal of Monetary Economics, 54(6): 1726--1768.

\bibitem{} Anderson, James E. and van Wincoop, Eric, 2004. ``Trade Costs.''\
Journal of Economic Literature, 42(3):\ 691--751.

\bibitem{} Arkolakis, Costas, Arnaud Costinot, and Andr\'{e}s Rodr\'{\i}%
guez-Clare, 2012. \textquotedblleft New Trade Models, Same Old
Gains?\textquotedblright \ American Economic Review, 102(1): 94--130.

\bibitem{} Armenter, Roc, Koren, Mikl\'{o}s, Menyh\'{e}rt, B\'{a}lint, Nagy, D\'{a}vid Kriszti\'{a}n, and Vuji\'{c}, Vukan, 2014. ``Using Post Office Opening Dates to Measure Historical Township Locations." Working paper.

\bibitem{} Baum-Snow, Nathaniel, Brandt, Loren, Henderson, J.\ Vernon,
Turner, Matthew A., and Zhang, Qinghua, 2012. ``Roads, Railroads and
Decentralization of Chinese Cities.''\ Working paper.

\bibitem{} Bleakley, Hoyt and Jeffrey Lin, 2012. \textquotedblleft Portage
and Path Dependence.''\ Quarterly Journal of Economics, 1--58.

\bibitem{} Co\c{s}ar, A. Kerem and Pablo D. Fajgelbaum, 2013.
\textquotedblleft Internal Geography, International Trade, and Regional
Outcomes.\textquotedblright \ Working paper.

\bibitem{} Dale, Frank T., 2003. \textquotedblleft Bridges over the Delaware
River: A History of Crossings.\textquotedblright \ Rutgers University Press.

\bibitem{} Desmet, Klaus and Rossi-Hansberg,\ Esteban, 2014. ``Spatial
Development.'' American Economic Review, 104(4): 1211--1243.

\bibitem{} Donaldson, Dave, 2012. ``Railroads of the Raj:\ Estimating the
Impact of Transportation Infrastructure.''\ American Economic Review,
forthcoming.

\bibitem{} Donaldson, Dave and Hornbeck, Richard, 2013. ``Railroads and
American Economic Growth:\ A\ `Market Access' Approach.'' Working paper.

\bibitem{} Duranton,\ Gilles, Morrow, Peter M., and Turner, Matthew A.,
2012. ``Urban Growth and Transportation.'' Review of Economic Studies, 79(4):
1407--1440.

\bibitem{} Fabinger, Michal, 2011. \textquotedblleft Trade and
Interdependence in a Spatially Complex World.\textquotedblright \ Working
paper.

\bibitem{} Feyrer, James, 2009a. ``Distance, Trade, and Income -- The 1967 to
1975 Closing of the Suez Canal as a Natural Experiment.''\ NBER\ Working
Paper 15557.

\bibitem{} Feyrer, James, 2009b. ``Trade and Income -- Exploiting Time Series
in Geography.'' NBER\ Working Paper 14910.

\bibitem{} GRUMP, 2000. Global Rural-Urban Mapping Project, Version 1
(GRUMPv1): Population Density Grid. Palisades, NY: Socioeconomic Data and
Applications Center (SEDAC), Columbia University.

\bibitem{} Rossi-Hansberg, Esteban, 2005. ``A\ Spatial Theory of Trade.''
American Economic Review, 95(5): 1464--1491.

\bibitem{} Tompsett, Anna, 2013. ``Bridges: Transport Infrastructure and 
Economic Geography on the Mississippi and Ohio 1860-2000.'' Working paper.
\end{thebibliography}

%TCIMACRO{%
%\TeXButton{appendix}{\appendix
%\section*{Appendix}}}%
%BeginExpansion
\appendix
\section*{Appendix}%
%EndExpansion

\section{Proofs}

We first state the following lemma that we use in the proofs of Propositions 
$1$ and $2$.

\begin{lemma}
Take two locations $\left( R,\ell ,h\right) $ and $\left( R,\ell ^{\prime
},h\right) $ such that $\ell ^{\prime }>\ell $.\ Then $p\left( R,\ell
^{\prime },h\right) \leq p\left( R,\ell ,h\right) $ if $R=H$, and $p\left(
R,\ell ^{\prime },h\right) \geq p\left( R,\ell ,h\right) $ if $R=F$.
\end{lemma}

\begin{proof}[Proof of Lemma 1]
We prove the lemma for $R=H$; the proof for $R=F$ involves the exact same
steps. Notice first that $p\left( H,\ell ,h\right) \geq p_{A}^{H}$ and $%
p\left( H,\ell ^{\prime },h\right) \geq p_{A}^{H}$ by the assumption that no
Home location specializes in good $M$. If $p\left( H,\ell ^{\prime
},h\right) =p_{A}^{H}$, then the result is immediate. So suppose $p\left(
H,\ell ^{\prime },h\right) >p_{A}^{H}$.\ Then $\left( H,\ell ^{\prime
},h\right) $ is fully specialized in $X$, and trades with the Foreign
region.\ As a consequence, there must exist a location at the river $\left(
H,0,\widehat{h}\right) $ such that $\left( H,\ell ^{\prime },h\right) $
trades through it.

Equilibrium condition (5) then implies%
\begin{equation*}
P_{X}\left( H,0,\widehat{h}\right) =e^{t_{X}d\left[ \left( H,0,\widehat{h}%
\right) ,\left( H,\ell ^{\prime },h\right) \right] }P_{X}\left( H,\ell
^{\prime },h\right) 
\end{equation*}%
and%
\begin{equation*}
P_{M}\left( H,0,\widehat{h}\right) =e^{-t_{M}d\left[ \left( H,0,\widehat{h}%
\right) ,\left( H,\ell ^{\prime },h\right) \right] }P_{M}\left( H,\ell
^{\prime },h\right) .
\end{equation*}%
Dividing these two equations yields%
\begin{equation}
p\left( H,0,\widehat{h}\right) =e^{\left( t_{X}+t_{M}\right) d\left[ \left(
H,0,\widehat{h}\right) ,\left( H,\ell ^{\prime },h\right) \right] }p\left(
H,\ell ^{\prime },h\right) .
\end{equation}

Similarly, by equilibrium condition (5),%
\begin{equation*}
P_{X}\left( H,0,\widehat{h}\right) \leq e^{t_{X}d\left[ \left( H,0,\widehat{h%
}\right) ,\left( H,\ell ,h\right) \right] }P_{X}\left( H,\ell ,h\right) 
\end{equation*}%
and%
\begin{equation*}
P_{M}\left( H,0,\widehat{h}\right) \leq e^{-t_{M}d\left[ \left( H,0,\widehat{%
h}\right) ,\left( H,\ell ,h\right) \right] }P_{M}\left( H,\ell ,h\right) ,
\end{equation*}%
irrespectively of whether $\left( H,0,\widehat{h}\right) $ and $\left(
H,\ell ,h\right) $ trade or not. Dividing the last two inequalities, we get%
\begin{eqnarray*}
p\left( H,0,\widehat{h}\right)  &\leq &e^{\left( t_{X}+t_{M}\right) d\left[
\left( H,0,\widehat{h}\right) ,\left( H,\ell ,h\right) \right] }p\left(
H,\ell ,h\right)  \\
&\leq &e^{\left( t_{X}+t_{M}\right) d\left[ \left( H,0,\widehat{h}\right)
,\left( H,\ell ^{\prime },h\right) \right] }p\left( H,\ell ,h\right) ,
\end{eqnarray*}%
where the second inequality follows from $\ell ^{\prime }>\ell $.\ Combining
this with equation (6) and cancelling $e^{\left( t_{X}+t_{M}\right) d\left[
\left( H,0,\widehat{h}\right) ,\left( H,\ell ^{\prime },h\right) \right] }$
on both sides yields the result.
\end{proof}

Now we are ready to prove Propositions $1$ to $3$.

\begin{proof}[Proof of Proposition 1]
Define%
\begin{equation*}
T^{R}=\left \{ \left( R,\ell ,h\right) :p\left( R,\ell ,h\right) \neq
p_{A}^{R}\right \} ,
\end{equation*}%
and%
\begin{equation*}
A^{R}=\left \{ \left( R,\ell ,h\right) :p\left( R,\ell ,h\right)
=p_{A}^{R}\right \} .
\end{equation*}

$R=T^{R}\cup A^{R}$ and $T^{R}\cap A^{R}$ follow directly from the
definitions.\ To see (i), notice that $p\left( R,\ell ,h\right) \neq
p_{A}^{R}$ necessarily implies that location $\left( R,\ell ,h\right) $ is
fully specialized, hence it trades with the other region.

For (ii), suppose that a location $\left( H,\ell ,h\right) $ from the
interior of $A^{H}$ is not in autarky, thus it trades with the Foreign
region. Then there must exist another location $\left( H,\ell ^{\prime
},h^{\prime }\right) \in A^{H}$ such that location $\left( H,\ell ,h\right) $
trades through it.\ By equilibrium condition (5), this implies%
\begin{equation*}
p\left( H,\ell ^{\prime },h^{\prime }\right) =p\left( H,\ell ,h\right)
e^{\left( t_{X}+t_{M}\right) d\left[ \left( H,\ell ,h\right) ,\left( H,\ell
^{\prime },h^{\prime }\right) \right] }>p\left( H,\ell ,h\right) ,
\end{equation*}%
which contradicts the fact that $p\left( H,\ell ^{\prime },h^{\prime
}\right) =p\left( H,\ell ,h\right) =p_{A}^{H}$.\ The argument is similar for 
$R=F$.

For (iii), we first prove that $p\left( R,\cdot ,\cdot \right) $ is a
continuous function. By equilibrium condition (5), we have%
\begin{equation*}
p\left( R,\ell ,h\right) \leq p\left( R,\ell ^{\prime },h^{\prime }\right)
e^{\left( t_{X}+t_{M}\right) d}
\end{equation*}%
and%
\begin{equation*}
p\left( R,\ell ^{\prime },h^{\prime }\right) \leq p\left( R,\ell ,h\right)
e^{\left( t_{X}+t_{M}\right) d}
\end{equation*}%
for any $\left( R,\ell ,h\right) $ and $\left( R,\ell ^{\prime },h^{\prime
}\right) $, where $d:=d\left[ \left( R,\ell ,h\right) ,\left( R,\ell
^{\prime },h^{\prime }\right) \right] $.\ Combining these two inequalities
yields%
\begin{equation*}
e^{-\left( t_{X}+t_{M}\right) d}\leq \frac{p\left( R,\ell ,h\right) }{%
p\left( R,\ell ^{\prime },h^{\prime }\right) }\leq e^{\left(
t_{X}+t_{M}\right) d}.
\end{equation*}%
Hence, in the limit as $\left( R,\ell ^{\prime },h^{\prime }\right)
\rightarrow \left( R,\ell ,h\right) $ (and thus $d\rightarrow 0$), we obtain 
$\frac{p\left( R,\ell ,h\right) }{p\left( R,\ell ^{\prime },h^{\prime
}\right) }\rightarrow 1$, that is, $p\left( R,\ell ^{\prime },h^{\prime
}\right) \rightarrow p\left( R,\ell ,h\right) $.\ This proves that $p\left(
R,\cdot ,\cdot \right) $ is continuous.

Now pick a location $\left( H,\ell ,h\right) \in A^{H}$ that is on the
boundary of $T^{H}$; the proof is similar for $R=F$. Clearly, location $%
\left( H,\ell ,h\right) $ weakly prefers autarky over trade as $p\left(
H,\ell ,h\right) =p_{A}^{H}$. Assume that $\left( H,\ell ,h\right) $
strictly prefers autarky over trade; this means $p\left( H,\ell ,h\right)
>p\left( H,\ell ^{\prime },h^{\prime }\right) e^{-\left( t_{X}+t_{M}\right) d%
\left[ \left( H,\ell ,h\right) ,\left( H,\ell ^{\prime },h^{\prime }\right) %
\right] }$ for all trading locations $\left( H,\ell ^{\prime },h^{\prime
}\right) \in T^{H}$. However, by the compactness and connectedness of $R$,
there exists a sequence of locations $\left \{ \left( H,\ell
_{m},h_{m}\right) \in T^{H}\right \} $ converging to $\left( H,\ell ,h\right) 
$.\ By continuity of $p\left( R,\cdot ,\cdot \right) $, there must exist a
large enough $m$ such that $p\left( H,\ell _{m},h_{m}\right) >p\left( H,\ell
^{\prime },h^{\prime }\right) e^{-\left( t_{X}+t_{M}\right) d\left[ \left(
H,\ell ,h\right) ,\left( H,\ell ^{\prime },h^{\prime }\right) \right] }$,
implying that $\left( H,\ell _{m},h_{m}\right) $ prefers autarky over
trade. But this contradicts the fact that $\left( H,\ell _{m},h_{m}\right)
\in T^{H}$.

For the final part, let $\widehat{\ell }\left( R,h\right) =\sup_{\ell
}\left \{ \left( R,\ell ,h\right) \in R:p\left( R,\ell ,h\right) \neq
p_{A}^{R}\right \} $ if there exists an $\ell $ such that $p\left( R,\ell
,h\right) \neq p_{A}^{H}$, and $\widehat{\ell }\left( R,h\right) =0$
otherwise.\ Then Lemma 1 implies that $p\left( H,\ell ,h\right) >p_{A}^{H}$
if $\ell <\widehat{\ell }\left( H,h\right) $, hence $\left( H,\ell ,h\right)
\in T^{H}$; and $p\left( H,\ell ,h\right) =p_{A}^{H}$ if $\ell >\widehat{%
\ell }\left( H,h\right) $, hence $\left( H,\ell ,h\right) \in A^{H}$.\ For
Foreign, $\ell <\widehat{\ell }\left( F,h\right) $ implies $p\left( F,\ell
,h\right) <p_{A}^{F}$, so $\left( F,\ell ,h\right) \in T^{F}$; and $\ell >%
\widehat{\ell }\left( F,h\right) $ implies $p\left( F,\ell ,h\right)
=p_{A}^{F}$, so $\left( F,\ell ,h\right) \in A^{F}$. This concludes the
proof.
\end{proof}

\begin{proof}[Proof of Proposition 2]
Average population density at distance $\ell $ from the river is%
\begin{equation*}
\int_{\underline{h}}^{\overline{h}}n\left( R,\ell ,h\right) dS\left( R,\ell
,h\right) ,
\end{equation*}%
and average population density at distance $\ell ^{\prime }$ is%
\begin{equation*}
\int_{\underline{h}}^{\overline{h}}n\left( R,\ell ^{\prime },h\right)
dS\left( R,\ell ^{\prime },h\right) .
\end{equation*}

Suppose $R=H$.\ Then, by Lemma 1, $p\left( R,\ell ,h\right) \geq p\left(
R,\ell ^{\prime },h\right) $, which, together with equation (3), implies $%
n\left( R,\ell ,h\right) \geq n\left( R,\ell ^{\prime },h\right) $ for all $%
h\in \left[ \underline{h},\overline{h}\right] $.\ As a consequence,%
\begin{equation*}
\int_{\underline{h}}^{\overline{h}}n\left( R,\ell ,h\right) dS\left( R,\ell
,h\right) \geq \int_{\underline{h}}^{\overline{h}}n\left( R,\ell ^{\prime
},h\right) dS\left( R,\ell ^{\prime },h\right) .
\end{equation*}

Now suppose $R=F$. Then Lemma 1 implies $p\left( R,\ell ,h\right) \leq
p\left( R,\ell ^{\prime },h\right) $, so by equation (4), $n\left( R,\ell
,h\right) \geq n\left( R,\ell ^{\prime },h\right) $ for all $h\in \left[ 
\underline{h},\overline{h}\right] $. Hence,%
\begin{equation*}
\int_{\underline{h}}^{\overline{h}}n\left( R,\ell ,h\right) dS\left( R,\ell
,h\right) \geq \int_{\underline{h}}^{\overline{h}}n\left( R,\ell ^{\prime
},h\right) dS\left( R,\ell ^{\prime },h\right) .
\end{equation*}
\end{proof}

\begin{proof}[Proof of Proposition 3]
If $R=H$, and $\left( R,\ell ,h\right) $ and $\left( R,\ell ^{\prime
},h^{\prime }\right) $ both trade over bridge $b$, then we have%
\begin{equation*}
p\left( R,0,h_{b}\right) =p\left( R,\ell ,h\right) e^{\left(
t_{X}+t_{M}\right) d\left[ \left( R,\ell ,h\right) ,\left( R,0,h_{b}\right) %
\right] }
\end{equation*}%
and%
\begin{equation*}
p\left( R,0,h_{b}\right) =p\left( T,\ell ^{\prime },h^{\prime }\right)
e^{\left( t_{X}+t_{M}\right) d\left[ \left( R,\ell ^{\prime },h^{\prime
}\right) ,\left( R,0,h_{b}\right) \right] }
\end{equation*}%
by equilibrium condition (5).

Then the fact that $\left( R,\ell ,h\right) $ is closer to the bridge than $%
\left( R,\ell ^{\prime },h^{\prime }\right) $ is equivalent to $p\left(
R,\ell ,h\right) >p\left( R,\ell ^{\prime },h^{\prime }\right) $, which, by
equation (3), is equivalent to $n\left( R,\ell ,h\right) >n\left( R,\ell
^{\prime },h^{\prime }\right) $.

If $R=F$, equilibrium condition (5) yields $p\left( R,\ell ,h\right)
<p\left( R,\ell ^{\prime },h^{\prime }\right) $ if and only if $\left(
R,\ell ,h\right) $ is closer to the bridge than $\left( R,\ell ^{\prime
},h^{\prime }\right) $, which is equivalent to $n\left( R,\ell ,h\right)
>n\left( R,\ell ^{\prime },h^{\prime }\right) $ by equation (4).
\end{proof}

\section{Additional figures}
% start from B1
\setcounter{figure}{0}
\renewcommand{\thefigure}{\thesection\arabic{figure}}

\dofigure{delaware_kernel}{The distribution of population densities near and far of bridges: Delaware}
\dofigure{hudson_kernel}{The distribution of population densities near and far of bridges: Hudson}
\dofigure{mississippi_kernel}{The distribution of population densities near and far of bridges: Mississippi}
\dofigure{missouri_kernel}{The distribution of population densities near and far of bridges: Missouri}
\dofigure{ohio_kernel}{The distribution of population densities near and far of bridges: Ohio}
\dofigure{tennessee_kernel}{The distribution of population densities near and far of bridges: Tennessee}



\end{document}
