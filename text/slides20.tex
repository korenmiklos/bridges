\documentclass[compress,mathserif]{beamer}
\usepackage[utf8]{inputenc}
%\usepackage[absolute]{textpos}
%\documentclass[handout,compress,mathserif]{beamer}
%\setbeameroption{show notes}

% This file is a solution template for:

% - Talk at a conference/colloquium.
% - Talk length is about 20min.
% - Style is ornate.



% Copyright 2004 by Till Tantau <tantau@users.sourceforge.net>.
%
% In principle, this file can be redistributed and/or modified under
% the terms of the GNU Public License, version 2.
%
% However, this file is supposed to be a template to be modified
% for your own needs. For this reason, if you use this file as a
% template and not specifically distribute it as part of a another
% package/program, I grant the extra permission to freely copy and
% modify this file as you see fit and even to delete this copyright
% notice.


\mode<presentation>
{
%  \usetheme{pittsburgh}
  % or ...

  \setbeamercovered{invisible}
  % or whatever (possibly just delete it)
}


\usepackage[USenglish]{babel}
\usepackage{ifthen,array}
\usepackage{amsthm}


\pretolerance5000 \hyphenpenalty9999
%\setlength{\TPHorizModule}{0.5cm} \setlength{\TPVertModule}{0.5cm}
%\textblockorigin{20mm}{20mm} % start everything near the top-left corner

\newcounter{ora}
\newcounter{perc}
\newcounter{kezdoora}
\newcounter{kezdoperc}
\newcounter{percek}
\setcounter{percek}{15}
\setcounter{kezdoora}{4} % for 1.35pm as the starting time

\providecommand{\leadingzero}[1]{\ifthenelse{\value{#1}<10}{0\arabic{#1}}{\arabic{#1}}}
\providecommand{\oradisplay}[1]{\ifthenelse{\value{#1}<60}{\arabic{kezdoora}:\leadingzero{#1}}{\setcounter{perc}{\value{#1}}\addtocounter{perc}{-60}\setcounter{ora}{\value{kezdoora}}\addtocounter{ora}{1}\arabic{ora}:\leadingzero{perc}}}

\providecommand{\notes}[1]{{\tiny\textbf{Note:} #1}}
%%%%%%%%%%%%%%%%%%%%%%%%%%%%%%%%%%%%%%%%%%%%%%%%
%% Hasznos matek makrok
%%%%%%%%%%%%%%%%%%%%%%%%%%%%%%%%%%%%%%%%%%%%%%%%

\newcommand{\QED}{{}\hfill$\Box$}
\newcommand{\intl}[4]{\int_{#1}^{#2} \! {#3} \, \mathrm d{#4}}
\newcommand{\period}{\text{.}} % Ez azert kell, mert a matek . mashogy nez ki, mint a szovege.
\newcommand{\comma}{\text{,}}  % Ez azert kell, mert a matek , mashogy nez ki, mint a szovege.
\newcommand{\dist}{\,\mathop{\operatorname{\sim\,}}\limits}
\newcommand{\D}{\,\mathop{\operatorname{d}}\!}
%\newcommand{\E}{\mathop{\operatorname{E}}\nolimits}
\newcommand{\Lag}{\mathop{\operatorname{L}}}
\newcommand{\plim}{\mathop{\operatorname{plim}}\limits_{T\to\infty}\,}
\newcommand{\CES}[3]{\mathop{\operatorname{CES}}\left(\left\{#1\right\},\left\{#2\right\},#3\right)}
\newcommand{\cestwo}[5]{\left[#1^\frac1{#5}\,#2^\frac{#5-1}{#5}+#3^\frac1{#5}\,#4^\frac{#5-1}{#5}\right]^\frac{#5}{#5-1}}
\newcommand{\cesmore}[4]{\left[\sum_{#3}#1_{#3}^\frac1{#4}\,{#2}_{#3}^\frac{#4-1}{#4}\right]^\frac{#4}{#4-1}}
\newcommand{\cesPtwo}[5]{\left[#1\,#2^{1-#5}+#3\,#4^{1-#5}\right]^\frac{1}{1-#5}}
\newcommand{\cesPmore}[4]{\left[\sum_{#3}#1_{#3}\,#2_{#3}^{1-#4}\right]^\frac{1}{1-#4}}
\newcommand{\diff}[2]{\frac{\D #1}{\D #2}}
\newcommand{\pdiff}[2]{\frac{\partial #1}{\partial #2}}
\newcommand{\convex}[2]{\lambda #1 + (1-\lambda)#2}
\newcommand{\ABS}[1]{\left| #1 \right|}
\newcommand{\suchthat}{:\hskip1em}
\newcommand{\dispfrac}[2]{\frac{\displaystyle #1}{\displaystyle #2}} % Emeletes tortekhez hasznos.

\newcommand{\diag}{\mathop{\mathrm{diag\mathstrut}}}
\newcommand{\tr}{\mathop{\mathrm{tr\mathstrut}}}
\newcommand{\E}{\mathop{\mathrm{E\mathstrut}}}
\newcommand{\Var}{\mathop{\mathrm{Var\mathstrut}}\nolimits}
\newcommand{\Cov}{\mathop{\mathrm{Cov\mathstrut}}}
\newcommand{\sgn}{\mathop{\operatorname{sgn\mathstrut}}}

\newcommand{\covmat}{\mathbf\Sigma}
\newcommand{\ones}{\mathbf 1}
\newcommand{\zeros}{\mathbf 0}
\newcommand{\BAR}[1]{\overline{#1}}

\renewcommand{\time}[1]{\addtocounter{percek}{#1}}

\newlength{\tempsep}

\newenvironment{subeqs}{\setlength{\tempsep}{\arraycolsep}
\setlength{\arraycolsep}{0.13889em} % Ez azert kell, hogy ne hagyjon tul sok helyet az = korul.
\begin{subequations}\begin{eqnarray}}
{\end{eqnarray}\end{subequations}
\setlength{\arraycolsep}{\tempsep}}

\newenvironment{tapad}{\setlength{\tempsep}{\arraycolsep}
\setlength{\arraycolsep}{0.13889em}} % Ez azert kell, hogy ne hagyjon tul sok helyet az = korul.
{\setlength{\arraycolsep}{\tempsep}}

\newenvironment{eqnarr}{\setlength{\tempsep}{\arraycolsep}
\setlength{\arraycolsep}{0.13889em} % Ez azert kell, hogy ne hagyjon tul sok helyet az = korul.
\begin{eqnarray}}
{\end{eqnarray} \setlength{\arraycolsep}{\tempsep}}

\newenvironment{eqnarr*}{\setlength{\tempsep}{\arraycolsep}
\setlength{\arraycolsep}{0.13889em} % Ez azert kell, hogy ne hagyjon tul sok helyet az = korul.
\begin{eqnarray*}}
{\end{eqnarray*} \setlength{\arraycolsep}{\tempsep}}


%\usepackage[active]{srcltx} % SRC Specials: DVI [Inverse] Search
% Fuzz --- -------------------------------------------------------
\hfuzz5pt % Don't bother to report over-full boxes < 5pt
\vfuzz5pt % Don't bother to report over-full boxes < 5pt
% THEOREMS -------------------------------------------------------
% MATH -----------------------------------------------------------
\newcommand{\norm}[1]{\left\Vert#1\right\Vert}
\newcommand{\abs}[1]{\left\vert#1\right\vert}
\newcommand{\set}[1]{\left\{#1\right\}}
\newcommand{\Real}{\mathbb R}
\newcommand{\eps}{\varepsilon}
\newcommand{\To}{\longrightarrow}
\newcommand{\BX}{\mathbf{B}(X)}
\newcommand{\A}{\mathcal{A}}




\newcommand{\directory}{exhibits}
\newcommand*{\newtitle}{\egroup\begin{frame}\frametitle}

\newcommand{\fullpagefigure}[2]{\begin{frame}\frametitle{\hyperlink{#1back}{#2}}\hypertarget{#1}{{\begin{center}\includegraphics[height=0.9\textheight]{\directory/#1}\end{center}}}\end{frame}}
\newcommand{\widefigure}[2]{\begin{frame}\frametitle{\hyperlink{#1back}{#2}}\hypertarget{#1}{{\begin{center}\includegraphics[width=\linewidth]{\directory/#1}\end{center}}}\end{frame}}
\newcommand{\longfigure}[2]{\begin{frame}\frametitle{\hyperlink{#1back}{#2}}\hypertarget{#1}{{\begin{center}\includegraphics[height=0.8\textheight]{\directory/#1}\end{center}}}\end{frame}}
\newcommand{\widetable}[2]{\begin{frame}\frametitle{\hyperlink{#1back}{#2}}\hypertarget{#1}{{\begin{center}\includegraphics[width=\linewidth]{tables/#1}\end{center}}}\end{frame}}
\newcommand{\longtable}[2]{\begin{frame}\frametitle{\hyperlink{#1back}{#2}}\hypertarget{#1}{{\begin{center}\includegraphics[height=0.8\textheight]{tables/#1}\end{center}}}\end{frame}}
\newcommand{\answer}[1]{\begin{itemize}\item #1\end{itemize}}


\newcommand{\jumpto}[2]{\hypertarget{#1back}{\hyperlink{#1}{#2}}}
\newcommand{\backto}[2]{\hypertarget{#1}{\hyperlink{#1back}{#2}}}

\begin{document}
\title{Bridges}
\author{Roc Armenter (Philly Fed)\\
Miklós Koren (CEU, KRTK, CEPR)\\
Dávid Krisztián Nagy (Princeton)}
\date{MKE 2013}
\maketitle

\section{Introduction}
\widefigure{upper-black-eddy}{The Upper Black Eddy Bridge on the Delaware River}

\begin{frame}
\frametitle{The Upper Black Eddy Bridge on the Delaware River}

\textquotedblleft \lbrack T]his new crossing brought additional business to
this part of the river valley. [...] By 1844, business in the now growing
town of Milford included three stores, three taverns, twelve to fiteen
mechanics' shops, a flour mill, and two new sawmills [...] Upper Black Eddy
on the Pennsylvania side of the river directly opposite Milford was a
favorite stop for timber raftsmen in the early days. By the mid-nineteenth
century the bridge brought even more business. Upper Black Eddy was booming,
too. It had forty houses, three hotels, and several stores and
shops.\textquotedblright \bigskip \newline
(Dale, F.:\ Bridges over the Delaware river. A history of crossings)

%TCIMACRO{\TeXButton{EndFrame}{\end{frame}}}%
%BeginExpansion
\end{frame}%
%EndExpansion
%TCIMACRO{\TeXButton{BeginFrame}{\begin{frame}}}%
%BeginExpansion
\begin{frame}%
%EndExpansion

\frametitle{Introduction}

\begin{itemize}
\item Build a theory to explain where people agglomerate in space.

\item Stylized but geographically rich environment:\ a subset of the globe,
divided by a river.

\begin{itemize}
\item Two banks differ in productivity $\Longrightarrow $ have an incentive
to trade.

\item Trade costly, both inland and across the river.

\item Crossing the river over a bridge is less costly than crossing it by
boat.
\end{itemize}

\item Building a bridge decreases the cost of crossing the river at a point $%
\Longrightarrow $ increases concentration of economic activity around that
point $\Longrightarrow $ brings about benefits from agglomeration.

\end{itemize}
\end{frame}


\begin{frame}\frametitle{Predictions}
\begin{itemize}
\item Model delivers a set of predictions on the spatial distribution of
population around rivers and bridges.

\begin{itemize}
\item Test predictions on high-resolution population data from the U.S.
\end{itemize}

\item Conduct two additional exercises:

\begin{itemize}
\item use historical data to check the effect of bridges on population
growth,

\item use model to measure the welfare gains from building a bridge.
\end{itemize}
\end{itemize}

%TCIMACRO{\TeXButton{EndFrame}{\end{frame}}}%
%BeginExpansion
\end{frame}%
%EndExpansion
%TCIMACRO{\TeXButton{BeginFrame}{\begin{frame}}}%
%BeginExpansion
\begin{frame}%
%EndExpansion

\frametitle{Related literature}

Economic geography models with continuous space.

\begin{itemize}
\item Two-good Ricardian models with agglomeration externalities:\
Rossi-Hansberg (2005), Desmet and Rossi-Hansberg (2012),\newline
Co\c{s}ar and Fajgelbaum (2012)

\item Neoclassical models with CES\ preferences:\newline
Fabinger (2011), Allen and Arkolakis (2013)\bigskip
\end{itemize}

Role of transport infrastructure in development.

\begin{itemize}
\item Railroads:\ Donaldson (2012), Donaldson and Hornbeck (2012)

\item Roads: Baum-Snow et al. (2012), Duranton et al. (2012)

\item Portage sites:\ Bleakley and Lin (2012)

\item Bridges on Mississippi and Ohio: Tompsett (2013)
\end{itemize}

%TCIMACRO{\TeXButton{EndFrame}{\end{frame}}}%
%BeginExpansion
\end{frame}%
%EndExpansion
%TCIMACRO{\TeXButton{BeginFrame}{\begin{frame}}}%
%BeginExpansion
\section{A spatial model with bridges}

\begin{frame}%
\frametitle{Geography}

\begin{itemize}
\item Continuous space. Two regions, $R\in \left \{ H,F\right \} $,
separated by a river without meanders.

\item A location is indexed by $\underset{\text{region}}{\underbrace{R}},%
\underset{\text{distance from river}}{\underbrace{\ell }},\underset{\text{%
closest river mile}}{\underbrace{h}}$,\newline
\newline
and is endowed with land $\lambda \left( R,\ell ,h\right) >0$.

\item Bridges at river miles $h_{1},\ldots ,h_{B}$.

\item Two goods, $i\in \left \{ X,M\right \} $.

Shipment of goods is costly:

\begin{itemize}
\item land shipping over distance $d$ involves iceberg cost $e^{t_{i}d}$,

\item crossing river by boat costs $e^{\tau _{i}^{0}}$,

\item crossing river over bridge $b$ costs $e^{\tau _{i}^{b}}<e^{\tau
_{i}^{0}}$.
\end{itemize}
\end{itemize}

\end{frame}%
%EndExpansion
%TCIMACRO{\TeXButton{BeginFrame}{\begin{frame}}}%
%BeginExpansion
\begin{frame}%
%EndExpansion

\frametitle{Consumption and production}
\begin{itemize}
\item Mass of $N$ consumers, freely mobile across space.

\item Consumer living at $\left( R,\ell ,h\right) $ owns local land,
supplies one unit of labor inelastically, and chooses her consumption by
maximizing%
\begin{equation*}
u\left( R,\ell ,h\right) =q_{X}\left( R,\ell ,h\right) ^{\frac{1}{2}%
}q_{M}\left( R,\ell ,h\right) ^{\frac{1}{2}}.
\end{equation*}

\item Good $i\in \left \{ X,M\right \} $ produced by perfectly competitive
firms combining labor and land under Cobb--Douglas production technology:%
\begin{equation*}
\underset{\text{output per unit of land}}{\underbrace{q_{i}\left( R,\ell
,h\right) }}=\left[ \underset{\text{unit cost of production}}{\underbrace{%
\overline{a}_{i}^{R}}}\right] ^{-1}\underset{\text{labor per unit of land}}{%
\cdot \left. \underbrace{n_{i}\left( R,\ell .h\right) }\right. ^{1-\alpha }}
\end{equation*}

\item Assume that Home has a \textit{comparative advantage} in $X$: $%
\overline{a}_{M}^{H}/\overline{a}_{X}^{H}$ high enough that no Home location
specializes in $M$. Similarly for Foreign.
\end{itemize}

%TCIMACRO{\TeXButton{EndFrame}{\end{frame}}}%
%BeginExpansion
\end{frame}%
%EndExpansion
%TCIMACRO{\TeXButton{BeginFrame}{\begin{frame}}}%
%BeginExpansion
\begin{frame}%
%EndExpansion

\frametitle{Two predictions}

%TCIMACRO{%
%\TeXButton{TeX field}{\begin{block}[Concentration at the river]}}%
%BeginExpansion
\begin{block}{Concentration at the river}
%EndExpansion
In equilibrium, average population density is decreasing in distance from
the river.%
%TCIMACRO{\TeXButton{TeX field}{\end{block}}}%
%BeginExpansion
\end{block}%
%EndExpansion

%TCIMACRO{%
%\TeXButton{TeX field}{\begin{block}[Concentration at bridges]}}%
%BeginExpansion
\begin{block}{Concentration at bridges}%
%EndExpansion
In any region $R$, take two locations $\left( R,\ell ,h\right) $ and $\left(
R,\ell ^{\prime },h^{\prime }\right) $ that trade over the same bridge.
Then, in equilibrium, $n\left( R,\ell ,h\right) >n\left( R,\ell ^{\prime
},h^{\prime }\right) $ iff $\left( R,\ell ,h\right) $ is closer to the
bridge than $\left( R,\ell ^{\prime },h^{\prime }\right) $.%
%TCIMACRO{\TeXButton{TeX field}{\end{block}}}%
%BeginExpansion
\end{block}%
%EndExpansion

%TCIMACRO{\TeXButton{EndFrame}{\end{frame}}}%
%BeginExpansion
\end{frame}%
%EndExpansion
%TCIMACRO{\TeXButton{BeginFrame}{\begin{frame}}}%
%BeginExpansion
\begin{frame}%
%EndExpansion

\frametitle{Two more predictions}

%TCIMACRO{%
%\TeXButton{TeX field}{\begin{block}[Left- and right-bank density positively correlated]}}%
%BeginExpansion
\begin{block}{Left- and right-bank density positively correlated}%
%EndExpansion
If the correlation between left- and right-bank log productivities is not
too low, then the correlation between left- and right-bank log population
densities is strictly positive.%
%TCIMACRO{\TeXButton{TeX field}{\end{block}}}%
%BeginExpansion
\end{block}%
%EndExpansion

%TCIMACRO{%
%\TeXButton{TeX field}{\begin{block}[Lower correlation at bridges]}}%
%BeginExpansion
\begin{block}{Lower correlation at bridges}%
%EndExpansion
The correlation between left- and right-bank log population densities is
lower at (trading) bridges than in general.%
%TCIMACRO{\TeXButton{TeX field}{\end{block}}}%
%BeginExpansion
\end{block}%
%EndExpansion

%TCIMACRO{\TeXButton{EndFrame}{\end{frame}}}%
%BeginExpansion
\end{frame}%
%EndExpansion
%TCIMACRO{\TeXButton{BeginFrame}{\begin{frame}}}%
%BeginExpansion
\begin{frame}%
%EndExpansion

\frametitle{Data}

Data on $12$ main rivers of the U.S.\ (Arkansas, Colorado, Columbia,
Connecticut, Delaware, Hudson, Mississippi, Missouri, Ohio, Red River,
Snake\ River, Tennessee).

\begin{itemize}
\item Geocoordinates of rivers come from ESRI\ Map of U.S. Major Waters.

\item Bridge data (location, year of construction) come from \textit{%
bridgehunter.com}.

\item Data on current population come from Version 1 of the Global
Rural-Urban Mapping Project population grid. (Historical data from HYDE.)

\begin{itemize}
\item Grid cells are $30\times 30$ arc seconds ($\approx 0.5\times 0.5$ km).
\end{itemize}
\end{itemize}

Create mapping between model and data:

\begin{itemize}
\item Measure $R$ (left or right bank), $\ell $ (distance from river), and $h
$\newline
(closest river mile) for each population grid cell.

\item Measure $h$ (river mile) for each bridge.
\end{itemize}

%TCIMACRO{\TeXButton{EndFrame}{\end{frame}}}%
%BeginExpansion
\end{frame}%
%EndExpansion
%TCIMACRO{\TeXButton{BeginFrame}{\begin{frame}}}%
%BeginExpansion


\widefigure{density-at-rivers}{Population density is higher at the river}
\widefigure{clustering-at-bridges}{Population \emph{clusters} at bridges}
\widefigure{correlation-of-two-sides}{Bridges increase correlation across sides}

\begin{frame}\frametitle{Population density before and after the bridge}
Log population density,
\[
\ln (N_{mt}/A_{mt}) = \mu_m+\nu_t+ \beta \text{bridge}_{mt}+e_{mt}.
\]

$m$: 5-mile sections of river

$t=1800, 1810, ..., 2000$
\end{frame}

\widefigure{regressions}{Population density before and after the bridges}

\begin{frame}\frametitle{Conclusion}

\begin{itemize}
\item 2D model of location and trade.
\item Spatial pattern of population around bridges is consistent with the model.
\item Bridge regions grow faster.
\item (Not shown today.) Potentially large welfare gains from early bridges.
\item (Todo.) Instrument for bridges.
\end{itemize}

\end{frame}
\end{document}
